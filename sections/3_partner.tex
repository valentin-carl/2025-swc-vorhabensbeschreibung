\clearpage
\section{Partners and Previous Work}
\label{sec:partner}

Das beschriebene Vorhaben ist in die Förderung im Software Campus mit Huawei Technologies Deutschland GmbH als Industriepartner sowie der Technischen Universität Berlin als akademischer Partner integriert.
Der Industriepartner wird durch das Vorhaben nicht gefördert, bietet jedoch ein paralleles Mentoring und wertvolle Perspektiven aus der Industrie.

\subsection{Research partner -- Technische Universität Berlin}
\label{sec:partner:tub}

Die Technische Universität Berlin ist eine der renommiertesten technischen Universitäten in Deutschland.
Mit einer breiten Palette von Studiengängen und einer starken Fokussierung auf Forschung und Innovation ist die Technische Universität Berlin eine der führenden Einrichtungen in Europa für technische Bildung und Wissenschaft.

Im Projekt wird die Technische Universität Berlin durch das Fachgebiet Mobile Cloud Computing vertreten, das 2017 gegründet wurde und Teil des Einstein Center Digital Future Berlin ist.
Unter der Leitung von Prof.~Dr.-Ing.~David Bermbach erforscht das Fachgebiet das Software-technische Design und die Experiment-getriebene Bewertung verteilter IT-Systeme im Kontext moderner Anwendungsdomänen.
Der Fokus liegt hier derzeit im Bereich der Datenmanagementsysteme und Anwendungsarchitekturen im Cloud-, Edge- und Fog-Computing, insbesondere bezogen auf Fragen des Placements von Daten und Anwendungskomponenten.

Wie in \cref{sec:standWiss} beschrieben konnte die Forschungsgruppe um Prof.~Dr.-Ing.~David Bermbach bereits zahlreiche Forschungsergebnisse zu Softwarearchitekturen und Anwendungsparadigmen für Edge und In-Network Computing in massiven LEO-Satellitennetzwerken auf international renommierten Konferenzen und Workshops veröffentlichen.
Als Teil des durch das BMBF geförderten Verbundprojektes \emph{6G NeXt} wird zudem bereits die Expertise des Fachgebiets zur Erforschung von Mobilfunk-, Telekommunikations- und Netzwerktechnologien der sechsten Generation (6G) beigetragen und ausgebaut.

\subsection{Industry partner -- Huawei Technologies Deutschland}
\label{sec:partner:industrie}

Huawei Technologies ist ein weltweit führender Anbieter von Informations- und Kommunikationstechnologie (IKT) mit Präsenz in über 170 Ländern und einem umfassenden Portfolio an Telekommunikationsprodukten und -lösungen.
In Deutschland ist Huawei seit 2001 aktiv und hat einen bedeutenden Einfluss auf die Wirtschaft und das Wachstum des Landes, indem es 2018 eine Bruttowertschöpfung von fast 2,3 Milliarden Euro und Beschäftigungseffekte für mehr als 28.000 Menschen generierte.
Huawei spielt eine entscheidende Rolle bei der Digitalisierung und der Einführung von Technologien wie 5G/6G, die die Grundlage für Smart Citys, Industrie 4.0 und nachhaltige Mobilität schaffen.

Für das Projekt bringt Huawei Technologies Deutschland entscheidende Erfahrungen in der Forschung und Entwicklung von 6G-Technologien und nicht-terrestrischen Netzwerken mit und kann zudem wertvolle Perspektiven aus Seiten der Industrie geben.
Besonders das Advanced Wireless Technologies Lab hat bereits signifikante Expertise in den Projekt-Kernforschungsbereichen Serverless"=Computing~\cite{10021037}, Edge- und In-Network-Computing~\cite{9996766} und nicht-terrestrischen und LEO-Satellitennetzwerken~\cite{10008593}.

\subsection{Relationship between research partner and industry partner}
\label{sec:partner:beziehung}

% welche Arbeitsbeziehungen / konkrete Unterstützung gibt es von Seiten des Unternehmens? Läuft das Projekt auf Vereinbarungen hinaus?

Im Rahmen des Projekts SPENCER gibt es eine enge Zusammenarbeit zwischen der Technischen Universität Berlin und Huawei Technologies Deutschland, insbesondere zwischen dem Fachgebiet Mobile Cloud Computing (Technische Universität Berlin) und dem Advanced Wireless Technologies Lab (Huawei Technologies Deutschland GmbH).
Diese Zusammenarbeit beinhaltet vordergründig einen bilateralen wissenschaftlichen und technischen Austausch über die gesamte Laufzeit des Projekts hinweg.
Die hohen Fachkompetenzen beider Partner steigern hier die Erfolgsaussichten für das Projekt:
Als Mentor und Ansprechpartner bei Huawei wird Dr.-Ing.~Osama Abboud, Senior Researcher des Advanced Wireless Technologies Lab am Huawei-Technologies-Standort München, die Anforderungsanalyse, Forschung und Entwicklung im Projekt aktiv mitgestalten und wertvolles Feedback geben, das zur kontinuierlichen Qualitätssicherung dient.
Als Leiter des Fachgebiets Mobile Cloud Computing und fachlicher Mentor des Mikroprojektleiters wird Prof.~Dr.-Ing.~David Bermbach die wissenschaftliche Qualitätssicherung übernehmen und die Umsetzung von Publikationen unterstützen.
