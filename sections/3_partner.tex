\clearpage
\section{Partners and Previous Work}
\label{sec:partner}

The project described is integrated into the Software Campus funding program with Huawei Technologies Deutschland GmbH as the industry partner and \TU\ as the academic partner.
The industry partner is not funded by the project but offers parallel mentoring and valuable perspectives from the industry.

\subsection{Research partner -- Technische Universität Berlin}
\label{sec:partner:tub}

\TU\ is one of the most renowned technical universities in Germany.
With a wide range of degree programs and a strong focus on research and innovation, \TU\ is one of Europe's leading institutions for technical education and science.

\TU\ is represented in the project by the Scalable Software Systems chair.
Under the direction of Prof.~Dr.-Ing.~Bermbach, the department researches the software design and experiment-driven evaluation of distributed IT systems in the context of modern application domains.
The current focus is on data management systems and application architectures in cloud, edge, and fog computing, particularly with regard to issues of data and application component placement.

The Scalable Software Systems group has already contributed numerous results in the areas of serverless computing, performance engineering, and edge computing.
Previously, the group successfully completed three other Software Campus projects (SPENCER, CODES, EMPIRIS), each addressing different aspects ranging from performance benchmarking to serverless architectures and applications in LEO satellite networks.
In addition, the chair has contributed expertise in networking as part of the BMBF-funded project 6G NeXt and is currently leading the DFG-funded project OptiFaaS, which, too, focuses on serverless computing.
This strong track record of projects and expertise directly underpins the objectives of the proposed GEKO project, providing an excellent foundation for advancing energy-efficient serverless platforms for AI inference.

\subsection{Industry partner -- Huawei Technologies Deutschland}
\label{sec:partner:industrie}

Huawei Technologies is a leading global provider of information and communications technology with a presence in over 170 countries and a comprehensive portfolio of telecommunications products and solutions.
Huawei has been active in Germany since 2001 and has had a significant impact on the country's economy and growth, generating gross value added of nearly €2.3 billion and employment effects for more than 28,000 people in 2018.
Huawei plays a crucial role in digitalization and the introduction of technologies such as 5G/6G, which form the basis for smart cities, Industry 4.0, and sustainable mobility.

For the GEKO project, Huawei Technologies Deutschland contributes extensive experience in the research and development of cloud and AI technologies and provides valuable industrial perspectives on building sustainable IT infrastructures.
In particular, the Advanced Wireless Technologies Lab has significant expertise in serverless computing, cloud and edge infrastructures, and resource-efficient platform architectures.
This expertise complements the scientific work of the GEKO project partners and supports the transfer of research results into practical, industry-relevant solutions.


\subsection{Relationship between research partner and industry partner}
\label{sec:partner:beziehung}

% welche Arbeitsbeziehungen / konkrete Unterstützung gibt es von Seiten des Unternehmens? Läuft das Projekt auf Vereinbarungen hinaus?

\TU\ and Huawei Technologies Deutschland already maintain a close and productive cooperation.
In 2025, both partners jointly established the Huawei Graduate School, which fosters long-term collaboration in research and education across cloud, edge, and AI technologies.
In addition, TU Berlin and Huawei have worked together within the Software Campus program on the project SPENCER, which investigated novel abstractions for edge computing in satellite networks.
These collaborations provide a strong foundation for extending the partnership within the GEKO project.

The GEKO project will involve close cooperation between \TU\ and Huawei Technologies Germany, particularly between the Scalable Software Systems chair (TUB) and the Advanced Wireless Technologies Lab (Huawei Technologies Germany GmbH).
This cooperation involves primarily a bilateral scientific and technical exchange throughout the entire duration of the project.

The high level of expertise of both partners ensures the project's chances of success:
As mentor and contact person at Huawei, Dr.~Sripriya Adhatarao, Senior Researcher at the Advanced Wireless Technologies Lab at Huawei Technologies in Munich, will actively participate in the requirements analysis, research, and development of the project and provide valuable feedback for continuous quality assurance.
As head of the Scalable Software Systems department and technical mentor to the micro-project manager, Prof.~Dr.-Ing.~Bermbach will be responsible for scientific quality assurance and will support the publication of findings.
