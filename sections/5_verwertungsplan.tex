\clearpage
\section{Utilization Plan}
\label{sec:verwertungsplan}
% Betrifft die Verwertung nach Projektende

This section outlines how the results of the GEKO project will be utilized, both in terms of industrial application and research impact, highlighting potential economic benefits, knowledge transfer, contributions to scientific advancement, and opportunities for follow-up research and innovation.

\subsection{Economic prospects of success}
\label{sec:verwertungsplan:wirtschaft}

% z.B. Open-Source-Stellung; produkt- oder dienstleistungsbezogen in Bezug auf Spin-Off oder ungeförderten Industriepartner (die FuE-Einrichtung selbst verwertet nicht wirtschaftlich)
% nicht der Fokus

%Obschon die wirtschaftliche Verwertung der im Projekt SPENCER gewonnen Erkenntnisse und Lö\-sung\-en nicht der Fokus des Projekts ist, bieten sich verschiedene Gelegenheiten zur industriellen Nutzbarkeit der Projektergebnisse.
%Zunächst werden jegliche im Rahmen des Projekts entstandenen Software-Artefakte als Open-Source-Software veröffentlicht (vgl.~AP3 Punkt 3.A in \cref{subsec:ap:3}).
%Dank einer Verwendung von Lizenzmodellen, die sowohl kommerzielle als auch nicht-kommerzielle Nutzung ermöglichen, steht diese Open-Source-Software der Allgemeinheit sowohl zur direkten Nutzung als auch zur Weiterentwicklung zur Verfügung.
%
%Auch der ungeförderte Industriepartner Huawei Technologies Deutschland GmbH kann jegliche Projektergebnisse in eigenen Produkten und Dienstleistungen nutzen.
%Zusätzlich profitiert Huawei Technologies Deutschland GmbH zudem durch den mit der Kooperation mit dem Fachgebiet Mobile Cloud Computing der Technischen Universität Berlin entstehenden Wissenstransfer.

Although the primary focus of the GEKO project is research rather than immediate commercialization, the outcomes nonetheless provide valuable opportunities for industrial utilization. All software artifacts developed during the project will be released as open-source under licensing models that permit both commercial and non-commercial use. This ensures that the broader community can not only directly adopt the developed serverless platform but also extend it further for domain-specific applications, thereby fostering innovation beyond the project itself.

The industry partner Huawei Technologies Deutschland GmbH can integrate the project results into its own products and services, particularly in the areas of sustainable cloud and edge computing infrastructures. In addition, Huawei benefits from the direct knowledge transfer enabled by the collaboration with the Scalable Software Systems group at Technische Universität Berlin, for instance through joint development of GPU orchestration strategies and evaluation methods. This exchange strengthens Huawei's capacity to bring more energy-efficient AI inference services into production environments while contributing to the competitiveness of the European and global ICT markets.

\subsection{Scientific and technical prospects of success}
\label{sec:verwertungsplan:wissenschaft}

% z.B. Know-how-Zuwachs, neue Forschungskontakte, Publikationen
% Qualifikationsarbeiten wie Bachelor/Master

The primary focus of the GEKO project is the scientific utilization of its results.
On the one hand, the project offers excellent opportunities for publishing its findings in internationally recognized scientific venues.
The project leadership's extensive track record of high-quality publications in the field of distributed and serverless systems underscores the potential for impactful dissemination.
Within the scope of GEKO, two publications are planned, for which the project lead will assume first authorship.
\needcheck{The specific conferences or journals for these publications are to be determined.}
Additionally, the project results will be documented in technical reports and shared through open-source repositories, further contributing to the research community and enabling follow-up work, in addition to the final report.

In addition, the GEKO project offers opportunities for students to gain qualifications.
As outlined in above, the student assistants are to remain involved in follow-up projects with the Scalable Software Systems group even after the project has ended and, depending on their qualifications, will have the opportunity to pursue a doctorate at \TU{}, for example.
The fundamentals of scientific work at PhD level are already being taught as part of the GEKO project.
In addition, the project can also provide a framework for qualification work by students at \TU\ at bachelor's and master's level, for example the supervision of final theses, as well as project work on topics whose results can be used scientifically.

\subsection{Scientific and economic connectivity}
\label{sec:verwertungsplan:anschluss}

% Weiterführende Arbeiten, neue Ideen, nächste Projekte

The GEKO project establishes a strong link between scientific research and industrial application by combining the Scalable Software Systems group's expertise in serverless computing with Huawei's experience in cloud and edge infrastructures.
Through the development of an open-source, energy-efficient serverless platform, knowledge and technical innovations are disseminated to both the research community and industry partners, enabling adoption, adaptation, and further experimentation.
The project results will be shared through publications, technical reports, and open-source releases, fostering collaboration and follow-up research in sustainable AI inference.
Looking forward, GEKO provides a foundation for future work, including extending the platform to additional AI workloads, exploring cross-cloud and hybrid deployments, investigating predictive scaling strategies, and evaluating long-term sustainability impacts.
These directions open opportunities for new research initiatives, collaborative projects, and potential commercial applications, ensuring that the scientific and economic benefits of the project continue to grow beyond its immediate duration.
