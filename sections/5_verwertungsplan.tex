\clearpage
\section{Utilization Plan}
\label{sec:verwertungsplan}
% Betrifft die Verwertung nach Projektende

In diesem Abschnitt werden die Chancen der Verwertung der entstandenen Erkenntnisse, Lösungen und Software umrissen.

\subsection{Economic prospects of success}
\label{sec:verwertungsplan:wirtschaft}

% z.B. Open-Source-Stellung; produkt- oder dienstleistungsbezogen in Bezug auf Spin-Off oder ungeförderten Industriepartner (die FuE-Einrichtung selbst verwertet nicht wirtschaftlich)
% nicht der Fokus

Obschon die wirtschaftliche Verwertung der im Projekt SPENCER gewonnen Erkenntnisse und Lö\-sung\-en nicht der Fokus des Projekts ist, bieten sich verschiedene Gelegenheiten zur industriellen Nutzbarkeit der Projektergebnisse.
Zunächst werden jegliche im Rahmen des Projekts entstandenen Software-Artefakte als Open-Source-Software veröffentlicht (vgl.~AP3 Punkt 3.A in \cref{subsec:ap:3}).
Dank einer Verwendung von Lizenzmodellen, die sowohl kommerzielle als auch nicht-kommerzielle Nutzung ermöglichen, steht diese Open-Source-Software der Allgemeinheit sowohl zur direkten Nutzung als auch zur Weiterentwicklung zur Verfügung.

Auch der ungeförderte Industriepartner Huawei Technologies Deutschland GmbH kann jegliche Projektergebnisse in eigenen Produkten und Dienstleistungen nutzen.
Zusätzlich profitiert Huawei Technologies Deutschland GmbH zudem durch den mit der Kooperation mit dem Fachgebiet Mobile Cloud Computing der Technischen Universität Berlin entstehenden Wissenstransfer.

\subsection{Scientific and technical prospects of success}
\label{sec:verwertungsplan:wissenschaft}

% z.B. Know-how-Zuwachs, neue Forschungskontakte, Publikationen
% Qualifikationsarbeiten wie Bachelor/Master

Im Vordergrund des Projekts SPENCER steht indes die wissenschaftliche Verwertung.
Diese Verwertung ist in zwei Dimensionen zu verstehen:
Zunächst bietet das Projekt exzellente Möglichkeiten zur Veröffentlichung von Projektergebnissen in international renommierten wissenschaftlichen Tagungen (vgl.~\cref{sec:finanz:reisen:konferenzen}).
Dies wird unter anderem durch die hohe Anzahl bereits durch die Projektleitung publizierter wissenschaftlicher Arbeiten von hoher Qualitätsgüte im Forschungsgebiet belegt (vgl.~\cref{sec:standWiss}).
Geplant sind insgesamt zwei Publikationen im Rahmen des Projekts, für die jeweils die Projektleitung die Erstautorschaft übernimmt.
Diese zwei Publikationen entfallen auf die internationalen wissenschaftlichen Konferenzen \emph{IEEE IC2E 2024} und \emph{ACM/IFIP Middleware 2024} (vgl.~\cref{sec:finanz:reisen:konferenzen}).
Zusätzlich wird ein Projektbericht als technischer Bericht ohne Qualitätssicherung veröffentlicht (vgl.~AP3 Punkt 3.B in \cref{subsec:ap:3})

Zusätzlich bietet das Projekt SPENCER hervorragende Möglichkeiten zur Qualifikation des wissenschaftlichen Nachwuchses.
Wie in \cref{sec:finanz:personal} umrissen, sollen die studentischen Beschäftigten auch nach Ablauf des Projekts in Anschlussprojekten an das Fachgebiet Mobile Cloud Computing gebunden werden und je nach Qualifikation beispielsweise Möglichkeit zur Promotion an der Technischen Universität Berlin erhalten (vgl.~auch \cref{sec:verwertungsplan:anschluss}).
Die Grundlagen des wissenschaftlichen Arbeitens auf Promotionsniveau werden dazu bereits im Rahmen des Projekts SPENCER vermittelt.
Darüber hinaus kann das Projekt auch den Rahmen für Qualifikationsarbeiten von Studierenden der Technischen Universität Berlin auf Bachelor- und Master-Niveau bieten, etwa indem Abschluss- und Projektarbeiten zu Projektthemen vergeben und betreut werden, deren Ergebnisse wissenschaftlich verwertet werden können.

\subsection{Scientific and economic connectivity}
\label{sec:verwertungsplan:anschluss}

% Weiterführende Arbeiten, neue Ideen, nächste Projekte
Das Forschungsgebiet massive Satellitennetzwerke ist, wie in \cref{sec:standWiss} beschrieben, noch ver\-hält\-nis\-mä\-ßig jung.
Das Projekt SPENCER ist eines der ersten Forschungsprojekte, das das spezifische Thema des Edge- und In-Network-Computing in massiven LEO-Satellitennetzwerken erst erschließen.
Damit bietet das Projekt eine vortreffliche Grundlage für Folgeprojekte, deren genaue Themen erst im Laufe der intensiven wissenschaftlichen Auseinandersetzung mit dem Themenkomplex im Rahmen des Projekts eruiert werden können.
Entsprechende Projekte ließen sich beispielsweise erneut im Rahmen des Programms Software Campus oder entsprechende Förderaufrufe zu Themengebieten wie Mobilfunktechnologien der sechsten Generation (6G) durch Bundesministerien fördern.
