\clearpage
\section{State of the Art in Science and Technology}
\label{sec:standWiss}

% - Nationale / internationale Arbeiten und die Unterschiede; eigene Vorarbeiten; (Abgrenzung der eigenen Forschungsarbeit muss deutlich werden)
% - Hinweis auf wichtigste aktuelle Projekte / Methoden (keine akademisch vollständigen Beschreibungen, Beschreibung muss zielbezogen sein, Referenzen möglich, aber nicht notwendig)
% - Abgrenzung zu aktuellen und laufenden Vorhaben mit Förderkatalog der Bundesregierung (\url{https://foerderportal.bund.de/foekat/jsp/StartAction.do}) und früheren SWC Projekten

Das Projekt SPENCER baut auf einer soliden Basis wissenschaftlicher Vorarbeiten einer kleinen aber aktiven Forschungsgemeinde im Bereich Edge"=Computing in Satellitennetzwerken auf.
Den Anstoß zur aktiven Erforschung von Softwarebereitstellung in Satellitennetzwerken gaben unter anderem Bhattacherjee et al.~\cite{Bhattacherjee2020-kr} mit einer Machbarkeitsstudie, in der die Möglichkeiten und Anwendungsfälle von Edge und In-Network Computing in massiven LEO-Satellitennetzwerken umrissen werden.
Zusätzlich beschrieben etwa Bhosale et al.~\cite{Bhosale2020-aa} mit \emph{Krios} ein erstes Anwendungsorganisationsparadigma für Edge und In-Network Computing in Satellitennetzwerken.
Krios basiert auf dem Container-Verwaltungssystem \emph{Kubernetes} und weist damit eine hohe Anzahl an Ineffizienzen und Problemen auf, die im Projekt SPENCER mit einem Serverless-Ansatz behoben werden.

Zudem ist das Projekt \emph{Tiansuan Constellation} der Beijing University of Posts and Telecommunications, Spacety und der Peking University hervorzuheben, in dem mit der Kombination industrieller und akademischer Ressourcen eine Open-Source Dienstplattform für Satelliten-Computing-Experimente im erdnahen Orbit entwickelt wird~\cite{wang2021tiansuan,wang2022tiansuan2,tiansuan2023online}.
Neben der Entwicklung von Anwendungen, Netzwerktechnologien und Software-Plattformen werden im Projekt auch eigene Satelliten entwickelt und bereitgestellt, auf denen Forschungsprototypen in realistischer Umgebung evaluiert werden können.

Neben diesen akademischen Arbeiten und Projekten beschäftigen sich bereits Akteure der Industrie mit Edge"=Computing im erdnahen Orbit.
Beispielsweise konnte Hewlett Packard Enterprise mit den Projekten \emph{HPE Spaceborne Computer} und \emph{HPE Spaceborne Computer 2} kommerzielle, frei verfügbare (\emph{commercial off-the-shelf}, COTS) Rechenserver auf der Internationalen Raumstation (\emph{International Space Station}, ISS) bereitstellen~\cite{hpespaceborne}.
Das Unternehmen OrbitsEdge stellte zudem das Produkt \emph{SATFRAME\textsuperscript{TM} 445 LE} vor, das Rahmenbedingungen für COTS-Rechen-Hardware im erdnahen Orbit bietet~\cite{orbitsedge2022}.

Die Übersicht über bestehende Arbeiten im Feld weist einen Mangel an Forschungsaktivitäten in Deutschland und Europa auf.
Hier kann auf Vorarbeiten der Projektleitung verwiesen werden, die im Rahmen des DFG-geförderten Projektes \emph{FogStore} am Fachgebiet Mobile Cloud Computing der Technischen Universität Berlin erbracht wurden.
Beispielsweise wurden bereits die bestehenden Forschungslücken in der Anwendungsorganisation für Edge und In-Network Computing in massiven LEO-Satellitennetzwerken dargelegt und vorgeschlagen, dass ein Serverless-Ansatz notwendig ist~\cite{paper_pfandzelter2021_LEO_serverless}.
Zudem konnten bereits erste Anwendungen für Edge"=Computing in Satellitennetzwerken vorschlagen werden~\cite{paper_pfandzelter2021_LEO_CDN,pfandzelter2023mars}.
Mit \emph{\textsc{Celestial}} wurde ein Open-Source Werkzeugset für die Experiment-getriebene Evaluierung von Softwaresystemen für Edge und In-Networking Computing in LEO-Satellitennetzwerken vorgestellt, das skalierbar und effizient Satelliten-Infrastruktur in der Cloud emuliert~\cite{pfandzelter2023celestialdemo,paper_becker2022_netem,pfandzelter2023failure,paper_pfandzelter2022_celestial,techreport_pfandzelter2022_celestial_extended}.
Es sei zudem auf Arbeiten des Projektleiters im Bereich Serverless"=Computing in Edge"=Computing verwiesen~\cite{paper_pfandzelter2020_tinyfaas,pfandzelter2023fred,pfandzelter2023enoki}.

Darüber hinaus gibt es Forschungsprojekte mit Förderung durch die Bundesregierung, die einen Bezug zu den Inhalten des Projekts SPENCER haben, besonders im Bereich Telekommunikationstechnologien der sechsten Generation (6G).
Beispielsweise untersuchen die Projekte \emph{6G-SKY}~\cite{6gsky} (Verbundprojekt gefördert durch das Bundesministerium für Wirtschaft und Klimaschutz, BMWK) und \emph{6G-TakeOff}~\cite{6gtakeoff} (Verbundprojekt gefördert durch das BMBF) holistische 3D-Netzwerkarchitekturen mit Satellitenkonnektivität.
Ähnlich entwickelt das Projekt \emph{Nitrides-4-6G}~\cite{nitrides46g} (Verbundprojekt gefördert durch das BMBF) Nitrid-basierte, dispersionsarme, effiziente Millimeterwellenbauelemente für strahlungsfeste Satelliten-Kommunikation.
Anders als das Projekt SPENCER kombinieren diese Projekte jedoch nicht Satellitennetzwerktechnik mit Edge und In-Network Computing.
Diese Bereiche werden eher in Projekten wie \emph{KeeCEK}~\cite{keecek} (Projekt gefördert durch das Bundesministerium für Digitales und Verkehr, BMDV), \emph{CLOUD56}~\cite{cloud56} und \emph{6G NeXt}~\cite{6gnext} (beide Verbundprojekte gefördert durch das BMBF).

Im aktuellen Stand der Wissenschaft und Technik bleiben entscheidene Forschungsfragen jedoch bisher unbeantwortet, die im Projekt SPENCER addressiert werden.
Es gibt bisher keine konkreten Vorschläge, welche Anwendungen sich für Edge-Computing in LEO-Satellitennetzwerken besonders eignen.
Besonders wichtig ist hier eine Definition der für die Anwendungen notwendigen Resourcen und Metriken, um eine objektive Bewertung von Ansätzen für Software- und Hardware-Plattformen für Edge-Computing in Satellitennetzwerken zu ermöglichen.
Darüber hinaus ist bisher wenig erforscht, wie von der hohen Netzwerkdynamik in erdnahen Satellitennetzwerken abstrahiert werden kann, da Edge-Anwendungen häufig ortsgebunden bereitgestellt werden sollen.
Schlussendlich ist bisher auch unklar, ob Serverless"=Computing für die geringen Resourcen in Edge-Infrastruktur auf Satelliten oder die Skalierbarkeit über tausende von Satelliten-Servern hinweg erweitert oder angepasst werden muss.
Diese Fragen stellen die Kernforschungsziele des Projekts SPENCER dar.
