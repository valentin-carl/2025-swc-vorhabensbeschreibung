\clearpage
\section{State of the Art in Science and Technology}
\label{sec:standWiss}

% - Nationale / internationale Arbeiten und die Unterschiede; eigene Vorarbeiten; (Abgrenzung der eigenen Forschungsarbeit muss deutlich werden)
% - Hinweis auf wichtigste aktuelle Projekte / Methoden (keine akademisch vollständigen Beschreibungen, Beschreibung muss zielbezogen sein, Referenzen möglich, aber nicht notwendig)
% - Abgrenzung zu aktuellen und laufenden Vorhaben mit Förderkatalog der Bundesregierung (\url{https://foerderportal.bund.de/foekat/jsp/StartAction.do}) und früheren SWC Projekten

\newcommand{\absatz}[1]{\emph{#1.}}

% overall: es gibt forschung über sustainable serverless computing => sharma
% wie ist FaaS grade?
% wird benutzt selber hosten und public cloud
% auf open-source-plattformen: k8s und ganz viel virtualisierung, energie-ineffizient! (overhead gvisor paper) 
% auf closed-source/public plattformen: kaum gpu support, was ai + serverless schwer macht
% aber: in der literatur einige ansätze wie man in beiden fällen mehr sustainability macht
    % was aber noch fehlt ist wie die serverless platform selber bestimmt, wie/wann/welche hardware benutzt wird
    % muss also noch alles zusammen gepackt werden

% 1 Satz was FaaS ist, funktionen erklären
% dann dass so halt ganz viele leute sich ressourcen teilen können
% warum ist faas cool? programmier abstraktion wodurch sich entwickler nicht um skalierbarkeit kümmern müssen und pay per use billing ...
% this model also has much untapped potential to be the key abstration for sustainable/green computing/ai
%The GEKO project builds on a solid foundation of scientific groundwork laid by a small but active research community in the field of sustainable serverless computing. % => auch schon im SWC gefördert
% sharma kurz sagen was die vision ist

% how is faas doing at the moment / in industry
% - either self hosted in open source platforms or public cloud which are opaque for researches
% - key characteristic: large scale resource sharing between many tenants
%   - hence isolation between functions is key!! now done at software-level but that comes with large costs (performance, efficiency, ...)
% - very energy-inefficient! but doesn't have to be like this forever
% - public cloud: wenig gpu support, dadurch ist es schwer, überhaupt alle möglichen arten von anwendungen damit zu realisieren
% - programming model also allows for areas where resources are extremely limited (e.g., leo edge)
% - open source mostly kubernetes-based, which also introduces large overheads and isn't really made for faas

% was macht die serverless forschung so
% - viel um performance (i.e. latenz) von einzelnen funktionsaufrufen zu verbessern
% - erste schrite die in richtung carbon accounting gehen, erste schritte mit overhead quantifizieren und einzelnen funktionen emissionen zuweisen (difficult technical challenge)
% - ganz viel scheduling, spatio-temportal, aber funktioniert nicht richtig: depending on the carbon intensity of the grid ... 
% - ganz viel kubernetes gedöns aber hat mega overhead wenn man energie effizienz machen will

% was für andere projekte gibt es von der bundesregierung?
% ???

% was ist die lücke die gecko füllt?
% - open source also nicht die probleme wie public faas
% - ermöglicht mehr anwendungen auf faas weil + gpu und deshalb auch AI
% - allerdings direkter fokus darauf, dass die serverless platform auch die hardware nutztung direkt bestimmt, i.e. kein mismatch dazwischen wofür systeme gebaut sind => bestmögliche ressourcennutzung





Function-as-a-Service is a serverless programming model in which developers express applications as small, stateless functions that are invoked on demand by the cloud platform. 
This abstraction frees developers from concerns about scalability and infrastructure management, while enabling large-scale resource sharing across many tenants. 
Combined with a pay-per-use billing model, FaaS has quickly become a central paradigm in both industry and research for building elastic applications~\cite{jonas_cloud_2019}.
Importantly, this model also has untapped potential as a cornerstone for sustainable computing: If orchestrated carefully, shared resources can be provisioned more efficiently at platform level than if each developer had to optimize for sustainability individually.
The GEKO project builds on a solid foundation of scientific groundwork laid by a small but active research community in the field of sustainable serverless computing. 
This includes early explorations into how serverless abstractions could become enablers for energy-aware resource management and carbon accounting. 
A cornerstone in this emerging area is the articulation of a broader vision for sustainable serverless computing, which frames FaaS as a potential driver of greener cloud services and highlights key research gaps that remain open~\cite{sharma_challenges_2023}.

In today's practice, however, serverless computing is far from energy efficient.
Current FaaS deployments follow two dominant paths. 
On the one hand, organizations operate open-source serverless platforms on top of Kubernetes or similar orchestration systems. 
While this approach offers flexibility, it suffers from heavy reliance on virtualization and container isolation, which introduces substantial overheads; these, in turn, increase per-request energy consumption between 15$\times$ and 30$\times$, depending on the virtualization technique used~\cite{sharma_challenges_2023}.
Recent studies show that software-based isolation layers such as gVisor can significantly degrade efficiency, limiting the potential for sustainable operation~\cite{young_true_nodate}. 
On the other hand, large public cloud providers offer commercial FaaS services, which remain largely opaque to researchers and offer only limited resource control (\needcheck{EXAMPLES}). 
In particular, GPU support is mostly absent from these platforms, making it difficult to realize AI workloads in a serverless fashion.
Despite these limitations, the research community has begun to address sustainability in serverless computing. 
Early work has explored quantifying the energy overheads of function isolation and assigning carbon intensity metrics to individual function invocations~\cite{sharma_accountable_2024}.
Other studies have examined spatio-temporal scheduling, where functions are steered to data centers with lower grid carbon intensity (\needcheck{EXAMPLES, viele}). 
However, these approaches often conflict with the latency requirements of serverless workloads and cannot fully exploit hardware-level optimizations~\cite{sukprasert_limitations_2024}. 
The majority of existing work still focuses on performance-oriented goals, such as reducing cold start latencies, rather than systematically reducing the energy footprint of the platform (\needcheck{EXAMPLES, viele}).
In the context of artificial intelligence, the gap is even more pronounced. 
Modern inference workloads are increasingly GPU-bound, but the lack of GPU integration in public serverless platforms is a contributor to preventing FaaS from being used for scalable AI inference. 
While research in AI sustainability has made progress on model-level and hardware-level optimizations (\needcheck{EXAMPLES}), the platform dimension, i.e., deciding how, when, and which hardware is activated for inference, remains largely unexplored. 
As a result, serverless computing today does not yet realize its potential as a key abstraction for sustainable AI.

Against this background, the GEKO project addresses a clear research gap. 
Unlike public FaaS platforms, GEKO builds on an open-source foundation that allows transparent investigation of GPU integration and orchestration. 
Unlike existing open-source solutions, it focuses not only on enabling GPU support but also on making the serverless platform itself responsible for hardware usage decisions. 
This system-level focus avoids the mismatch between application-level assumptions and platform-level realities, ensuring that resources are shared and utilized as efficiently as possible. 
As a result, GEKO directly advances the state of the art in sustainable FaaS and establishes the groundwork for scalable, energy-efficient AI inference.
