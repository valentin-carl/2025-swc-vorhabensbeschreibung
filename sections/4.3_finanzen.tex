\subsection{Financial planning}
\label{sec:finanz}

% Finanzierungsplan bzw. Vorkalkulationm, nach bestem Wissen; Grundsatz eines sparsamen und angemessenen Mitteleinsatzes, Ausgaben- bzw. Kostenpositionen detailliert darstellen und nachvollziehbar begründen

% Kategorien zur Verwendung von Mitteln.

%     Personal (0812, 0822),
%     Gegenstände bis zu 800 Euro (0831)
%     Mieten und Rechnerkosten (0834)
%     Vergabe von Aufträgen (0835)
%     Sonstige allgemeine Verwaltungsausgaben (bspw. Literatur) (0843)
%     Dienstreisen (0846)
%     Gegenstände und andere Investitionen $\ge$ 800 Euro
%     (0850)




\begin{table}[]
    % \resizebox{\linewidth}{!}{
    \renewcommand{\arraystretch}{1.3}
    \centering
    \begin{tabular}{lllrr}
        \toprule
        \textbf{Finanzposten} & \textbf{Positions-Nr.} & \textbf{Einzelposten}                                                                                        & \textbf{Ausgaben}     \\
        Personal              & 0822                   & \begin{tabular}[c]{@{}l@{}}Studentische Hilfskraft \'a 80 Monatsstunden\\(TV Stud \textrm{III})\end{tabular} & 14.400,00~€           \\
                              & 0822                   & \begin{tabular}[c]{@{}l@{}}Studentische Hilfskraft \'a 80 Monatsstunden\\(TV Stud \textrm{III})\end{tabular} & 14.400,00~€           \\
                              & 0822                   & \begin{tabular}[c]{@{}l@{}}Studentische Hilfskraft \'a 80 Monatsstunden\\(TV Stud \textrm{III})\end{tabular} & 14.400,00~€           \\
                              & 0822                   & \begin{tabular}[c]{@{}l@{}}Studentische Hilfskraft \'a 80 Monatsstunden\\(TV Stud \textrm{III})\end{tabular} & 14.400,00~€           \\
                              & 0822                   & \begin{tabular}[c]{@{}l@{}}Studentische Hilfskraft \'a 80 Monatsstunden\\(TV Stud \textrm{III})\end{tabular} & 14.400,00~€           \\
                              & 0822                   & \begin{tabular}[c]{@{}l@{}}Studentische Hilfskraft \'a 80 Monatsstunden\\(TV Stud \textrm{III})\end{tabular} & 14.400,00~€           \\
                              & 0822                   & \begin{tabular}[c]{@{}l@{}}Studentische Hilfskraft \'a 80 Monatsstunden\\(TV Stud \textrm{III})\end{tabular} & 14.400,00~€           \\
                              &                        & \textbf{$\sum$ Personalausgaben}                                                                             & \textbf{100.800,00~€} \\
        \midrule
        Dienstreisen          & 0846                   & Software Campus                                                                                              & 1.720,00~€            \\
                              & 0846                   & Industriepartner                                                                                             & 3.776,00~€            \\
                              & 0846                   & Konferenzen                                                                                                  & 4.630,00~€            \\
                              &                        & \textbf{$\sum$ Dienstreisen}                                                                                 & \textbf{10.126,00~€}  \\
        \midrule
                              &                        & \textbf{$\sum$ Ausgaben}                                                                                     & \textbf{110.926,00~€} \\
        \bottomrule
    \end{tabular}
    % }
    \caption{Übersicht der Finanzplanung}
    \label{tab:finanzen}
\end{table}


Basierend auf der Arbeits- und Ressourcenplanung in \crefrange{subsec:ap}{subsec:zeitplan} stellen wir nun die finanzielle Planung des Projekts SPENCER vor.
Dies stellt eine Vorkalkulation nach bestem Wissen und Gewissen dar, der der Grundsatz eines sparsamen und angemessenen Mitteleinsatzes zugrunde liegt.

Eine Übersicht der Finanzplanung kann \cref{tab:finanzen} entnommen werden.
Es liegen nur die Finanzposten Personal (\cref{sec:finanz:personal}) und Dienstreisen (\cref{sec:finanz:reisen}) vor.
Es sind keine Fremdaufträge für Forschung und Entwicklung geplant.
Es wird eine Gesamtsumme von 110.926,00~€ beantragt.

\subsubsection{Personalausgaben}
\label{sec:finanz:personal}

Wie in \cref{subsec:zeitplan} beschrieben werden sieben studentische Hilfskräfte mit der inhaltlichen Bearbeitung des Projekts betraut und erhalten dafür je eine Anstellung über 80 Monatsstunden nach dem an der Technischen Universität Berlin geltenden Tarifvertrages für studentische Beschäftigte \textrm{III}.
% Auf Grundlage des angespannten Personalmarktes, besonders im Fachbereich Informatik in der Metropolregion Berlin, soll zudem jeder/m Beschäftigten eine Zulage von 50\% gemäß der Regelungen der Technischen Universität Berlin gewährt werden.
% Eine Besetzung der für dieses Projekt beantragten Stellen ohne eine solche Zulage ist nicht absehbar.
% Auf Grundlage der Empfehlungen der Technischen Universität Berlin zur finanziellen Absicherung des Projekts wird darüber hinaus ein weiterer Aufschlag von 5\% berechnet.
Die Personalausgaben belaufen sich somit auf 14.400,00~€ pro studentischer Hilfskraft pro Jahr.
Dies deckt jegliche Lohnausgaben und Steuern des Beschäftigungsverhältnisses ab und enthält sowohl den Rentenversicherungsbeitrag als auch den gesetzlichen Beitrag zur Unfallkasse Berlin.
In Summe belaufen sich die Personalausgaben damit auf 100.800,00~€.
Es sei an dieser Stelle darauf hingewiesen, dass die Einstellung von studentischen Hilfskräften der Stärkung der wissenschaftlichen Erfolgsaussichten dienen soll.
Je nach Qualifikationsstand sollen alle Mitarbeiter:innen im Rahmen von Folgeprojekten an das Fachgebiet Mobile Cloud Computing gebunden werden, beispielsweise durch die Möglichkeit zur Promotion (vgl.~\cref{sec:verwertungsplan:anschluss}).
Die technische Ausstattung der studentischen Beschäftigten wird durch das Fachgebiet Mobile Cloud Computing der Technischen Universität Berlin gestellt.

\subsubsection{Dienstreisen}
\label{sec:finanz:reisen}

% Übernachtungskosten nach ARVVwV 2023
% https://www.static.tu.berlin/fileadmin/www/10004170/Rechtsgrundlagen/arvvwv_2023_pdf.pdf
% Reisekostensätze Inland: https://www.tms.bund.de/Webs/TMS/DE/Gesetze/Reisekosten/Reisekostensaetze-Inland/reisekostensaetze-inland_node.html
% Workshop: 3 Tage mit Abreise
% Konferenz: 7 Tage mit Abreise
% SWC: 2 Tage mit Abreise

\begin{table}[]
    \renewcommand{\arraystretch}{1.3}
    \centering
    \resizebox{\linewidth}{!}{
        \begin{tabular}{lllrrrrrr}
            \toprule
            \textbf{Bezeichnung}                  & \textbf{Ziel}        & \textbf{Kategorie} & \begin{tabular}[c]{@{}l@{}}\textbf{An-/Abreise}\end{tabular} & \begin{tabular}[c]{@{}l@{}}\textbf{Gebühr}\end{tabular} & \begin{tabular}[c]{@{}l@{}}\textbf{Hotel}\end{tabular} & \begin{tabular}[c]{@{}l@{}}\textbf{Tagegeld}\end{tabular} & \begin{tabular}[c]{@{}l@{}}\textbf{Dauer}\end{tabular} & \begin{tabular}[c]{@{}l@{}}\textbf{Gesamt}\end{tabular} \\ \midrule
            SWC Summit                            & Deutschland          & Software Campus    & 100,00~€                                                     & --                                                      & 80,00~€                                                & 28,00~€                                                   & 3 Tage                                                 & 344,00~€                                                \\
            SWC Training 1                        & Deutschland          & Software Campus    & 100,00~€                                                     & --                                                      & 80,00~€                                                & 28,00~€                                                   & 3 Tage                                                 & 344,00~€                                                \\
            SWC Training 2                        & Deutschland          & Software Campus    & 100,00~€                                                     & --                                                      & 80,00~€                                                & 28,00~€                                                   & 3 Tage                                                 & 344,00~€                                                \\
            SWC Training 3                        & Deutschland          & Software Campus    & 100,00~€                                                     & --                                                      & 80,00~€                                                & 28,00~€                                                   & 3 Tage                                                 & 344,00~€                                                \\
            SWC Training 4                        & Deutschland          & Software Campus    & 100,00~€                                                     & --                                                      & 80,00~€                                                & 28,00~€                                                   & 3 Tage                                                 & 344,00~€                                                \\
            Besuch Industriepartner Q2 (PL)       & München, Deutschland & Industriepartner   & 100,00~€                                                     & --                                                      & 80,00~€                                                & 28,00~€                                                   & 2 Tage                                                 & 236,00~€                                                \\
            Besuch Industriepartner Q2 (Stud. HK) & München, Deutschland & Industriepartner   & 100,00~€                                                     & --                                                      & 80,00~€                                                & 28,00~€                                                   & 2 Tage                                                 & 236,00~€                                                \\
            Besuch Industriepartner Q2 (Stud. HK) & München, Deutschland & Industriepartner   & 100,00~€                                                     & --                                                      & 80,00~€                                                & 28,00~€                                                   & 2 Tage                                                 & 236,00~€                                                \\
            Besuch Industriepartner Q2 (Stud. HK) & München, Deutschland & Industriepartner   & 100,00~€                                                     & --                                                      & 80,00~€                                                & 28,00~€                                                   & 2 Tage                                                 & 236,00~€                                                \\
            Besuch Industriepartner Q2 (Stud. HK) & München, Deutschland & Industriepartner   & 100,00~€                                                     & --                                                      & 80,00~€                                                & 28,00~€                                                   & 2 Tage                                                 & 236,00~€                                                \\
            Besuch Industriepartner Q2 (Stud. HK) & München, Deutschland & Industriepartner   & 100,00~€                                                     & --                                                      & 80,00~€                                                & 28,00~€                                                   & 2 Tage                                                 & 236,00~€                                                \\
            Besuch Industriepartner Q2 (Stud. HK) & München, Deutschland & Industriepartner   & 100,00~€                                                     & --                                                      & 80,00~€                                                & 28,00~€                                                   & 2 Tage                                                 & 236,00~€                                                \\
            Besuch Industriepartner Q2 (Stud. HK) & München, Deutschland & Industriepartner   & 100,00~€                                                     & --                                                      & 80,00~€                                                & 28,00~€                                                   & 2 Tage                                                 & 236,00~€                                                \\
            Besuch Industriepartner Q4 (PL)       & München, Deutschland & Industriepartner   & 100,00~€                                                     & --                                                      & 80,00~€                                                & 28,00~€                                                   & 2 Tage                                                 & 236,00~€                                                \\
            Besuch Industriepartner Q4 (Stud. HK) & München, Deutschland & Industriepartner   & 100,00~€                                                     & --                                                      & 80,00~€                                                & 28,00~€                                                   & 2 Tage                                                 & 236,00~€                                                \\
            Besuch Industriepartner Q4 (Stud. HK) & München, Deutschland & Industriepartner   & 100,00~€                                                     & --                                                      & 80,00~€                                                & 28,00~€                                                   & 2 Tage                                                 & 236,00~€                                                \\
            Besuch Industriepartner Q4 (Stud. HK) & München, Deutschland & Industriepartner   & 100,00~€                                                     & --                                                      & 80,00~€                                                & 28,00~€                                                   & 2 Tage                                                 & 236,00~€                                                \\
            Besuch Industriepartner Q4 (Stud. HK) & München, Deutschland & Industriepartner   & 100,00~€                                                     & --                                                      & 80,00~€                                                & 28,00~€                                                   & 2 Tage                                                 & 236,00~€                                                \\
            Besuch Industriepartner Q4 (Stud. HK) & München, Deutschland & Industriepartner   & 100,00~€                                                     & --                                                      & 80,00~€                                                & 28,00~€                                                   & 2 Tage                                                 & 236,00~€                                                \\
            Besuch Industriepartner Q4 (Stud. HK) & München, Deutschland & Industriepartner   & 100,00~€                                                     & --                                                      & 80,00~€                                                & 28,00~€                                                   & 2 Tage                                                 & 236,00~€                                                \\
            Besuch Industriepartner Q4 (Stud. HK) & München, Deutschland & Industriepartner   & 100,00~€                                                     & --                                                      & 80,00~€                                                & 28,00~€                                                   & 2 Tage                                                 & 236,00~€                                                \\
            IEEE IC2E 2024                        & Paphos, Zypern       & Konferenz          & 415,00~€                                                     & 850,00~€                                                & 125,00~€                                               & 35,00~€                                                   & 6 Tage                                                 & 2.100,00~€                                              \\
            ACM/IFIP Middleware 2024              & Hongkong, SAR, China & Konferenz          & 483,00~€                                                     & 750,00~€                                                & 145,00~€                                               & 61,00~€                                                   & 7 Tage                                                 & 2.530,00~€                                              \\
            \midrule
            \textbf{$\sum$}                       &                      & Software Campus    &                                                              &                                                         &                                                        &                                                           &                                                        & 1.720,00~€                                              \\
                                                  &                      & Industriepartner   &                                                              &                                                         &                                                        &                                                           &                                                        & 3.776,00~€                                              \\
                                                  &                      & Konferenz          &                                                              &                                                         &                                                        &                                                           &                                                        & 4.630,00~€                                              \\
            \textbf{$\sum$}                       &                      &                    &                                                              &                                                         &                                                        &                                                           &                                                        & \textbf{10.126,00~€}                                    \\
            \bottomrule
        \end{tabular}
    }
    \caption{Übersicht der geplanten Reisen und deren zugehöriger Ausgaben}
    \label{tab:reisen}
\end{table}


Obschon digitale Kommunikationsformate, etwa E-Mail, Chat oder Videotelefonie-Schalten, flexible Möglichkeiten zum Projekt-bezogenen Austausch der Stakeholder bieten, sind Treffen in Person dem Erfolg des Projekts unerlässlich.
Dies erfordert regelmäßige Dienstreisen der mit dem Projekt betrauten Mitarbeiter:innen.
Konkret beinhaltet dies drei Arten von Reisen, deren Finanzplanung in diesem Abschnitt genauer beleuchtet wird: Summits und Trainings im Rahmen des Software Campus (\cref{sec:finanz:reisen:swc}), Treffen mit dem Industriepartner (\cref{sec:finanz:reisen:industriepartner}) und der Besuch internationaler wissenschaftlicher Konferenzen (\cref{sec:finanz:reisen:konferenzen}).
Eine Übersicht über die Finanzplanung für Dienstreisen ist \cref{tab:reisen} zu entnehmen.

\paragraph{Software-Campus-Summit und -Trainings}
\label{sec:finanz:reisen:swc}

Für Projektleiter:innen von Mikroprojekten im Rahmen des Software Campus ist die Teilnahme an einem `Summit' im Jahr und insgesamt vier Trainings verpflichtend.
Diese Veranstaltungen finden an verschiedenen, bisher jedoch noch nicht bekannten Orten in Deutschland statt und werden jeweils zwei Übernachtungen in Anspruch nehmen.
Für die Teilnahme an diesen Veranstaltungen durch die Projektleitung wird jeweils eine An- und Abreise von insgesamt 100,00~€ veranschlagt, etwa mit der Deutschen Bahn.
Zudem werden auf Grundlage der Empfehlungen und Regeln der Technischen Universität Berlin pauschal Übernachtungsausgaben (etwa in Hotels) von 80,00~€ pro Nacht kalkuliert.
Auf Grundlage des Bundesreisekostengesetzes in aktueller Fassung wird ein Tagegeld von 28,00~€ geplant.
Es fallen keine Teilnahmegebühren an.

\paragraph{Industriepartner-Besuche}
\label{sec:finanz:reisen:industriepartner}

Über die Laufzeit des Projektes von einem Jahr sind zwei Besuche beim Industriepartner Huawei Technologies Deutschland in München geplant, an dem Projektleitung und studentische Beschäftigte teilnehmen, um die aktuelle Projektlage in einem Intensiv-Workshop zu besprechen und weitere Schritte zu planen.
Diese Besuche dienen auch der Dissemination von Projektergebnissen innerhalb der Organisation des Industriepartners, etwa mit Vorträgen.
Ähnlich zu den Veranstaltungen, deren Finanzplanung in \cref{sec:finanz:reisen:swc} umrissen ist, werden pauschale An- und Abreiseausgaben von 100,00~€ pro Person pro Besuch veranschlagt.
Zudem werden erneut pauschale Übernachtungsausgaben von 80,00~€ pro Nacht und Tagegeld von 28,00~€ berechnet, indes ist nur eine Dauer von zwei Tagen (eine Übernachtung) pro Besuch geplant.
Es fallen keine Teilnahmegebühren an.

\paragraph{Besuch von wissenschaftlichen Konferenzen und Workshops}
\label{sec:finanz:reisen:konferenzen}

In der Informatik im Allgemeinen und im Forschungsgebiet Rechnersysteme und verteilte Systeme im Besonderen stellen Veröffentlichungen auf internationalen Konferenzen und Workshops den primären Publikationsweg dar, da sie eine zeitgemäße Dissemination von Forschungsergebnissen ermöglichen.
Für die Veröffentlichung von Publikationen in Tagungsbänden von Konferenzen und Workshops ist der Besuch dieser Veranstaltungen in Verbindung mit einem Vortrag über die Forschungsergebnisse verpflichtend.
Traditionell wird diese Aufgabe durch die/den Erstautor/in einer Publikation übernommen.
Basierend auf der bisherigen Publikationsrate des Projektleiters (vgl.~\cref{sec:standWiss,sec:partner:tub}) werden für das Projekt SPENCER insgesamt zwei Besuche von Konferenzen veranschlagt.
Es sei darauf hingewiesen, dass international anerkannte Tagungen nur international abgehalten werden.

Die geplanten Konferenzen stellen international anerkannte Tagungen dar, auf denen bereits Vorarbeiten der Projektleitung zum Projektthema vorgestellt wurden (vgl.~\cref{sec:standWiss}).
Die in \cref{tab:reisen} angeführten An- und Abreiseausgaben basieren auf aktuellen Angeboten, die in \cref{sec:anhang:reisen} aufgeführt sind.
Es wird mit zeitgemäßer An- und Abreise einen Tag vor respektive einen Tag nach der Veranstaltung kalkuliert.
Die angeführten Tagungsgebühren beziehen sich auf Gebühren der Tagungsinstanzen 2023.
Die kalkulierten Übernachtungsausgaben beziehen sich basierend auf den Empfehlungen und Vorschriften der Technischen Universität Berlin auf die \emph{Allgemeine Verwaltungsvorschrift über die Neufestsetzung der Auslandstage- und Auslandsübernachtungsgelder vom 13.10.2022} (ARVVwV) und stellen somit erwartbare Höchstausgaben dar.
Es kann nicht ausgeschlossen werden, dass spezielle Hotelangebote für Tagungsteilnehmende angeboten werden, die zum jetzigen Zeitpunkt jedoch noch nicht vorliegen.
Auch das veranschlagte Tagegeld bezieht sich auf die ARVVwV in aktueller Fassung.

Die \emph{IEEE International Conference on Cloud Engineering} (IEEE IC2E) ist eine bedeutende Veranstaltung für Forschende, die sich mit Cloud-Computing und verwandten Themen befassen.
Die Konferenz konzentriert sich auf die neuesten Trends in Cloud-Engineering, einschließlich Cloud-Architekturen, Dienstgüte, Sicherheit und Skalierbarkeit.
Die IEEE IC2E bietet eine Plattform für den Wissensaustausch und die Diskussion über bewährte Praktiken, um die Leistung und Zuverlässigkeit von Cloud-Infrastrukturen zu verbessern.
Die IEEE IC2E 2024 findet vom 24.~bis 27.~September 2024 in Paphos, Zypern statt (vgl.~\href{https://conferences.computer.org/IC2E/2024/}{Konferenzwebsite}).

Die \emph{ACM/IFIP International Middleware Conference} (ACM/IFIP Middleware) ist eine führende Konferenz im Bereich Middleware-Technologien und -systeme.
Die Veranstaltung bringt Wissenschaftler:innen, Ingenieur:innen und Industrieexpert:innen zusammen, um aktuelle Entwicklungen in den Bereichen Software-Middleware, verteilte Systeme und Dienstorientierte Architekturen zu präsentieren und zu diskutieren.
Middleware spielt eine entscheidende Rolle bei der Integration von Anwendungen und Diensten, weshalb die Konferenz einen bedeutenden Beitrag zur Entwicklung von Middleware-Lösungen leistet.
Die ACM/IFIP Middleware 2024 findet vom 11.~bis 15.~Dezember 2024 in Hongkong, SAR, China statt (vgl.~\href{https://middleware-conf.github.io/2024/}{Konferenzwebsite}).

% \subsubsection{Cloud-Rechenkapazität}
% \label{sec:finanz:cloud}

% Für die agile Entwicklung des Software-Prototypen und dessen rigide Evaluierung (vgl.~AP2 in \cref{subsec:ap}) wird Rechenkapazität jenseits der durch mobile Workstations der mit der inhaltlichen Bearbeitung des Mikroprojekts betrauten Mitarbeiter:innen abgedeckten Kapazität nötig sein.
% Konkret soll dieser Befarf durch flexibles Mieten von Rechenkapazität eines Cloud-Anbieters gedeckt werden, da dies ein agiles Umsetzen der Arbeitspakete ermöglicht.
% Beispielsweise kann auf Rechenkapazität während der Bearbeitung von AP1 im Wesentlichen verzichtet werden, während sie zur Fertigstellung von AP2 in erhöhtem Umfang unerlässlich ist.

% \begin{table}[]
    \renewcommand{\arraystretch}{1.3}
    \centering
    \resizebox{\linewidth}{!}{
        \begin{tabular}{lrrrrrr}
            \toprule
            \textbf{Instanztyp}     & \begin{tabular}[c]{@{}c@{}}\textbf{Anzahl}\end{tabular} & \begin{tabular}[c]{@{}c@{}}\textbf{Einzelansatz/Stunde}\end{tabular} & \begin{tabular}[c]{@{}c@{}}\textbf{Stunden/Arbeitstag}\end{tabular} & \begin{tabular}[c]{@{}c@{}}\textbf{Arbeitstage/Monat}\end{tabular} & \begin{tabular}[c]{@{}c@{}}\textbf{Zeitplan AP2 Punkt 2.B}\end{tabular} & \begin{tabular}[c]{@{}c@{}}\textbf{Gesamt}\end{tabular} \\ \midrule
            \texttt{n2-standard-32} & 6 Instanzen                                             & $\sim$1.974~€                                                        & 4 Stunden                                                           & 20 Tage                                                            & 7 Monate                                                                & 6.632,36~€                                              \\
            \midrule
            \textbf{$\sum$}         &                                                         &                                                                      &                                                                     &                                                                    &                                                                         & \textbf{6.632,36~€}                                     \\
            \bottomrule
        \end{tabular}
    }
    \caption{Übersicht der Ausgaben für Cloud-Rechenkapazität (vgl.~\cref{sec:anhang:cloud})}
    \label{tab:cloud}
\end{table}


% Es bietet sich hier die Cloud-Platform \emph{Google Cloud Platform} an, deren Programmierschnittstellen bereits mit dem in \cref{sec:standWiss} vorgestellten Evaluationswerkzeug \emph{\textsc{Celestial}} integriert sind.
% Basierend auf der Arbeit an diesem Werkzeug werden für die Entwicklung und Evaluierung des Software-Prototyps in AP2 etwa sechs Maschinen des Typs \texttt{n2-standard-32} mit 32 vCPUs, 128GB Arbeitsspeicher und 50GB Festplatte in der Cloud-Region Frankfurt veranschlagt.
% Diese Maschinen kosten jeweils etwa 1,974~€ pro Stunde (vgl.~\cref{sec:anhang:cloud}).
% Bei einer Laufzeit von etwa vier Stunden am Arbeitstag über die für Punkt 2.B von AP2 (Prototypische Entwicklung Software-Plattform) veranschlagte Laufzeit von sieben Monaten ergeben sich Kosten von insgesamt 6.632,36~€.
% Die zugrundeliegende Kalkulation ist \cref{tab:cloud} zu entnehmen.

% \subsubsection{Demonstrator für Wissenschaftskommunikation}
% \label{sec:finanz:demonstrator}

% Für die in AP3 Punkt 3.B geplante Wissenschaftskommunikation, in der die Forschungsfragen und -ergebnisse des Mikroprojekts einer breiten Öffentlichkeit im Rahmen von Veranstaltungen zugänglich gemacht werden sollen, wird ein Demonstrator unerlässlich sein.
% Die bisherige Erfahrung in der Kommunikation um das Forschungsfeld Satellitennetzwerke hat gezeigt, dass Attribute wie Satellitenorbits, Skalen von massiven LEO-Satellitennetzwerken und deren Dynamizität wenig bis nicht intuitiv für viele Menschen ist, besonders wenn diese keine Vorkenntnisse mitbringen.
% Entsprechend wird für die Unterstützung der Umsetzung von AP3 Punkt 3.B ein Demonstrator geplant, der etwa im Rahmen der Langen Nacht der Wissenschaften in Berlin, an der das Fachgebiet Mobile Cloud Computing der Technischen Universität Berlin im Kontext des Einstein Center Digital Future Berlin seit seiner Gründung teilnimmt, ausgestellt werden kann.

% Dieser Demonstrator ist lediglich ein geeignet großes Public-Info-Display, das durch einen Slim-PC angesteuert werden kann.
% Auf diesem Display, das auf einem Standfuß angebracht werden soll um für Interessierte auf Augenhöhe zugänglich zu sein, werden etwa im Projekt entwickelte Animationen oder Präsentationen gezeigt, die die Forschungsprobleme und -ergebnisse veranschaulichen.
% Es werden für ein \emph{Samsung QB65B} als geeignet großes Public-Info-Display 795,89~€, für einen \emph{myWall HT21} Standfuß 84,80~€ und für ein \emph{Lenovo ThinkCentre M60e Tiny 11LV0099GE} Slim-PC 529,99~€ veranschlagt (vgl.~Vergleichsangebote in \cref{sec:anhang:demonstrator}).
% Damit belaufen sich die Gesamtkosten für den Demonstrator auf 1.410,68~€.
